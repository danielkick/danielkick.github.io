%%%%%%%%%%%%%%%%%%%%%%%%%%%%%%%%%%%%%%%%%
% Twenty Seconds Resume/CV
% LaTeX Template
% Version 1.1 (8/1/17)
%
% This template has been downloaded from:
% http://www.LaTeXTemplates.com
%
% Original author:
% Carmine Spagnuolo (cspagnuolo@unisa.it) with major modifications by
% Vel (vel@LaTeXTemplates.com)
%
% Adapted to be an Rmarkdown template by Rob J Hyndman
% 23 November 2018
%
% License:
% The MIT License (see included LICENSE file)
%
%%%%%%%%%%%%%%%%%%%%%%%%%%%%%%%%%%%%%%%%%

%----------------------------------------------------------------------------------------
%	PACKAGES AND OTHER DOCUMENT CONFIGURATIONS
%----------------------------------------------------------------------------------------

\documentclass[10pt,a4paper,]{twentysecondcv}

\usepackage[scaled=0.86]{DejaVuSansMono}

\usepackage{ifxetex,ifluatex}
\usepackage{fixltx2e} % provides \textsubscript
\ifnum 0\ifxetex 1\fi\ifluatex 1\fi=0 % if pdftex
  \usepackage[T1]{fontenc}
  \usepackage[utf8]{inputenc}
  \usepackage{csquotes}

\setlength{\emergencystretch}{3em}  % prevent overfull lines
\providecommand{\tightlist}{%
  \setlength{\itemsep}{0pt}\setlength{\parskip}{0pt}}
\setcounter{secnumdepth}{0}

\usepackage{booktabs}



%----------------------------------------------------------------------------------------
%	 PERSONAL INFORMATION
%----------------------------------------------------------------------------------------

% If you don't need one or more of the below, just remove the content leaving the command, e.g. \cvnumberphone{}

% Profile pic
\profilepic{logo.png}

\cvname{Daniel R. Kick PhD} % Your name

\cvjobtitle{Research Geneticst} % Job title/career
\cvdate{July 2022} % Date of birth or date of CV??
\cvaddress{Room 213, Curtis Hall, University of Missouri-Columbia
Columbia, MO
65211} % Short address/location, use \newline if more than 1 line is required
\cvnumberphone{} % Phone number
\cvsite{\href{http://orcid.org/0000-0002-9002-1862}{orcid.org/0000-0002-9002-1862}} % Personal website
\cvmail{\href{mailto:daniel.r.kick@protonmail.com}{\nolinkurl{daniel.r.kick@protonmail.com}}} % Email address
\cvtwitter{}
\cvgithub{\href{https://github.com/danielkick}{danielkick}}
\cvlinkedin{\href{https://www.linkedin.com/in/\url{https://www.linkedin.com/in/daniel-kick-5a449b9a/}}{\url{https://www.linkedin.com/in/daniel-kick-5a449b9a/}}}

\lfoot{\sf Curriculum Vitae: Daniel R. Kick PhD}
\rfoot{\sf\thepage}

% Pandoc CSL macros
\newlength{\cslhangindent}
\setlength{\cslhangindent}{1.5em}
\newlength{\csllabelwidth}
\setlength{\csllabelwidth}{3em}
\newenvironment{CSLReferences}[3] % #1 hanging-ident, #2 entry spacing
 {% don't indent paragraphs
  \setlength{\parindent}{0pt}
  % turn on hanging indent if param 1 is 1
  \ifodd #1 \everypar{\setlength{\hangindent}{\cslhangindent}}\ignorespaces\fi
  % set entry spacing
  \ifnum #2 > 0
  \setlength{\parskip}{#2\baselineskip}
  \fi
 }%
 {}
\usepackage{calc}
\newcommand{\CSLBlock}[1]{#1\hfill\break}
\newcommand{\CSLLeftMargin}[1]{\parbox[t]{\csllabelwidth}{#1}}
\newcommand{\CSLRightInline}[1]{\parbox[t]{\linewidth - \csllabelwidth}{#1}}
\newcommand{\CSLIndent}[1]{\hspace{\cslhangindent}#1}

\begin{document}
\aboutme{}

% Skill bar section, each skill must have a value between 0 an 6 (float)
%\skills{{pursuer of rabbits/5.8},{good manners/4},{outgoing/4.3},{polite/4},{Java/0.01}}

% Skill text section, each skill must have a value between 0 an 6
%\skillstext{{lovely/4},{narcissistic/3}}

\makeprofile % Print the sidebar




\begin{itemize}
\tightlist
\item
  \texttt{profilepic}: A local file path to an image
\item
  \texttt{aboutme}: A short description that is included in a template
  specific location
\item
  \texttt{headcolor}: A featured colour for the template
\end{itemize}

\hypertarget{in-brief}{%
\section{In Brief:}\label{in-brief}}

\begin{itemize}
\tightlist
\item
  I'm interested in applying statistical and computational tools to
  better predict and explain biological effects. I aim to use these
  techniques to improve food security by assisting in generation of
  cultivates with high yield that are robust to variable weather.
\item
  During graduate school I focused on connecting ion channel mRNA
  abundances to electrophysiological properties and homeostatic tuning
  within neurons.
\item
  Now, as a Post-Doc with the USDA-ARS I focus on using deep learning to
  improve yield prediction by better representing interaction effects
  between genetic, enviromental, and management effects.
\end{itemize}

\hypertarget{selected-skills}{%
\section{Selected Skills}\label{selected-skills}}

\begin{itemize}
\item
  R (Tidyverse, Shiny (see
  \href{https://github.com/danielkick/MultiMethodPhysApp}{source},
  \href{https://danielkick.shinyapps.io/BioSc3700_1/}{deployed})
\item
  Python (Pandas, Numpy, Plotly)
\item
  Bash, git, Vim, slurm, basic singularity, basic docker
\item
  Statistics (Parametric, basic Bayesian modeling, Resampling)
\item
  Machine Learning, Deep Learning (Keras, Scikit-learn, Caret)
\item
  Electrophysiology (voltage, current, and dynamic clamp),
  Microdissection
\item
  Scientific communication, Scientific writing, Figure generation
  (Inkscape)
\end{itemize}

\hypertarget{professional-experience}{%
\section{Professional Experience}\label{professional-experience}}

\nopagebreak

\begin{twenty}
    \twentyitem{2021-Present}{Research Geneticist}{USDA-ARS}{Responsibilities include: Aggregating, cleaning, and imputing public datasets, tuning and training deep learning models using Keras, assisting an undergraduate run high throughput phenotyping project, and producing scripts to streamline the same. This is a post-doctoral research position overseen by Dr. Jacob D. Washburn.\par\empty}
\end{twenty}

\hypertarget{education}{%
\section{Education}\label{education}}

\nopagebreak

\begin{twenty}
    \twentyitem{2021 Ph.D.}{University of Missouri, Columbia, MO}{Coursework included: Grant Writing, Quantitative Methods in the Life Sciences, and Machine Learning Methods for Biomedical Informatics}{(3.97/4 GPA)\par\empty}
    \twentyitem{2015 B.S.}{Truman State University, Kirksville, MO}{Coursework included: Next Generation Sequence Data and Analysis, Bioinformatics, Analysis of Variance and Experimental Design, Non-Parametric Statistics, and Economic \& Medicinal Botany.}{(3.65/4 GPA)\par\empty}
\end{twenty}

\hypertarget{short-courses-and-workshops}{%
\section{Short Courses and
Workshops}\label{short-courses-and-workshops}}

\nopagebreak

\begin{twenty}
    \twentyitem{2020}{Software Carpentry: Python}{University of Missouri, Columbia, MO}{Two-day workshop providing hands on experience working with Python, Git, and Unix Shell.\par\empty}
    \twentyitem{2017}{Diversity \& Inclusion Workshop}{University of Missouri, Columbia, MO}{Six-hour workshop covering implicit bias and inclusive practices.\par\empty}
    \twentyitem{2016}{Big Data in Biology}{University of Texas, Austin, TX}{Weeklong short courses in bioinformatics: Introduction to Core NGS Tools and Introduction to RNA-Seq\par\empty}
\end{twenty}

\hypertarget{research-experience}{%
\section{Research Experience}\label{research-experience}}

\nopagebreak

\begin{twenty}
    \twentyitem{2016—2021}{PhD Candidate}{Dr. David Schulz (University of Missouri)}{Analyzed the efficacy of machine learning methods to reproduce neuron cell types given mRNA abundances. (see Northcutt et al. 2019). • Described regulation of gap junction conductance controlled by timing of activity in the Cancer borealis cardiac ganglion. • Contrasted responses in cell activity, membrane currents, and channel mRNA abundances in response to potassium channel blockade after one or twenty-four hours in the Cancer borealis cardiac ganglion.\par\empty}
    \twentyitem{2015}{PhD Student}{Dr. Lorin Milescu (University of Missouri)}{• Devised and tested a protocol for live imaging Drosophila with a two-photon microscope.\par\empty}
    \twentyitem{2015}{PhD Student}{Dr. Bing Zhang (University of Missouri)}{• Designed a machine to assay Drosophila climbing (used in Willenbrink et al. 2016) and quantified sleep patterns of Drosophila tyrosine hydroxylase mutants.\par\empty}
    \twentyitem{2014-2015}{Undergraduate Research Assistant}{Dr. Diane Janick-Buckner \& Dr. Brent Buckner (Truman State University) }{• Prototyped a hydroponic growth chamber for root morphology characterization of maize mutants.\par\empty}
    \twentyitem{2014}{REU Student}{Dr. Rahul Kanadia (University of Connecticut)}{ • Conduced a pilot study using in situ hybridization which implicated upregulation of the minor spliceosome in postponing retinal cell death. \par\empty}
    \twentyitem{2013}{REU Student}{Dr. Christian Lorsen (University of Missouri)}{ • Measured motor function decline in a mouse model of spinal muscular atrophy receiving an oligonucleotide treatment.\par\empty}
    \twentyitem{2011}{Student Volunteer}{Dr. Laszlo Kovacs (Missouri State University)}{• Refined a set of teaching labs to be used in a genetics course.\par\empty}
\end{twenty}

\hypertarget{teaching-experience}{%
\section{Teaching Experience}\label{teaching-experience}}

\begin{itemize}
\item
  Teaching Assistant (Animal Physiology Lab, University of Missouri)
\item
  Oversaw and coordinated transition of the lab to a hybrid then fully
  online class model due to Covid-19.
\item
  Adapted and expanded materials to be suitable for remote learning.
\item
  Constructed and deployed a statistics web tool (source, deployed).
  2018 -- 2021 Curriculum Consultant (Animal Physiology Lab, University
  of Missouri) 2018
\item
  Updated curriculum and redesigned experiments placing a greater focus
  primary literature and data analysis.
\end{itemize}

Teaching Assistant (Animal Physiology Lab, University of Missouri) •
Assisted students through experiments. • Provided short weekly lectures.
2015 -- 2016''

\hypertarget{honors-and-awards}{%
\section{Honors and Awards}\label{honors-and-awards}}

\nopagebreak

\begin{twenty}
    \twentyitem{2019}{J. Perry Gustafson Award}{University of Missouri }{\$2000\par\empty}
    \twentyitem{2016 –2018}{NIH T32 Training Grant Recipient}{}{\$27,000 yearly\par\empty}
    \twentyitem{2017}{NIH T32 Travel Award}{}{\$750\par\empty}
    \twentyitem{2016}{NIH T32 Travel Award}{}{\$750\par\empty}
    \twentyitem{2015}{Graduated Cum Laude}{Truman State University}{\empty\empty}
\end{twenty}

\hypertarget{outreach-and-communication}{%
\section{Outreach and Communication}\label{outreach-and-communication}}

\nopagebreak

\begin{twenty}
    \twentyitem{2022}{Tools and Techniques for a Jupyter Based Scientific Workflow}{Invited Workshop, Bioinformatics in Plant Science (BIPS)}{University of Missouri\par\empty}
    \twentyitem{2019}{Can mRNA expression recapitulate neuron cell types?}{Alumni Research Presentation}{Truman State University\par\empty}
    \twentyitem{2019}{Spare the synapse, spoil the circuit}{Public presentation, Science on Tap}{Columbia Missouri\par\empty}
    \twentyitem{2019}{Voltage Dependent modification of Electrical Synapses}{Biological Sciences Recruitment Poster Session}{University of Missouri\par\empty}
    \twentyitem{2018}{Gap Junction Conductance Modulation Via Voltage}{Alumni Research Presentation}{Truman State University\par\empty}
    \twentyitem{2017}{Please mind the gap: Network homeostatic plasticity in the Cancer borealis cardiac ganglion}{Alumni Research Presentation}{Truman State University\par\empty}
    \twentyitem{2016}{The Tell-Tale Heart: Applying crustacean neurogenic hearts to basic neurosciences questions}{Alumni Research Presentation}{Truman State University\par\empty}
    \twentyitem{2016}{Scientific Poster Judge}{Spring Undergraduate Research and Creative Achievements Forum}{University of Missouri\par\empty}
\end{twenty}

\hypertarget{mentoring}{%
\section{Mentoring}\label{mentoring}}

\nopagebreak

\begin{twenty}
    \twentyitem{2017 – 2019}{Undergraduate Student}{Abby Beckerdite}{\empty\empty}
    \twentyitem{2018 – 2018}{NSF REU Student}{Ayla Ross}{\empty\empty}
    \twentyitem{2017 – 2017}{NSF REU Student}{Katlyn Sullivan}{\empty\empty}
    \twentyitem{2016 – 2016}{Post-baccalaureate Scholar}{Rody Kingston}{\empty\empty}
    \twentyitem{2016 – 2016}{NSF REU Student}{Kelly Hiersche}{\empty\empty}
\end{twenty}

\hypertarget{publications}{%
\section{Publications}\label{publications}}


\end{document}
